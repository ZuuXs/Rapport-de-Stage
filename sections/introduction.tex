Dans le cadre d’un processus de modernisation de la gestion des services informatiques, et plus spécifiquement pour assurer la conformité et l’optimisation des environnements Microsoft, j’ai eu l’opportunité de développer un outil d’audit des tenants Microsoft au sein de \textbf{Exakis Nelite}. Ce rapport de stage présente les différentes étapes de cette mission, depuis la conception jusqu’à la réalisation de l’outil.

En effet, avec l'évolution rapide des technologies et l'importance croissante de la sécurité informatique, il est devenu impératif pour les entreprises de s'assurer que leurs systèmes sont non seulement efficaces mais aussi sécurisés. L’objectif fondamental de ce projet est de fournir un outil qui permet de vérifier et d’optimiser les configurations des tenants Microsoft, en se basant sur les standards et les meilleures pratiques en vigueur.

Actuellement, les entreprises utilisent divers outils pour gérer leurs environnements Microsoft, mais un besoin crucial se fait sentir pour un audit automatisé, systématique et efficace qui peut être effectué en temps réel. Les indicateurs fournis par les différents systèmes doivent être vérifiés et comparés aux standards en vigueur pour garantir une gestion optimale. C’est ainsi que le développement de cet outil d’audit constitue une nécessité pertinente et opportune pour améliorer le suivi et l’optimisation des environnements Microsoft.

Ce rapport est structuré de la manière suivante : le chapitre I est dédié à la présentation de l’organisme d’accueil. Le chapitre II traite du cadrage fonctionnel. Le chapitre III est consacré au cadrage technique et enfin le chapitre IV présente les réalisations.