La réalisation de ce stage au sein de l’Université Paris Est Créteil (\textbf{UPEC}) et de l’entreprise Exakis Nelite (\textbf{EXN}) m’a permis de développer des compétences techniques et professionnelles significatives. L'objectif principal de ce stage est de concevoir et de déployer une interface web pour l’audit et la gestion des environnements Microsoft 365 et Azure. 

Pendant ce projet, j’ai eu l’opportunité de travailler avec diverses technologies avancées, notamment Microsoft Azure pour la configuration des ressources et le déploiement d’applications, PowerShell pour l’automatisation des tâches, C\# pour le développement d’applications, Microsoft 365 pour la gestion de services comme SharePoint, Teams, et OneDrive, ainsi que DevOps pour la mise en place de pipelines CI/CD. Ces compétences seront des atouts précieux pour ma carrière future.

Les principales réalisations de ce stage incluent l’analyse des besoins, la compréhension des exigences et des objectifs du projet, le développement d’un Proof of Concept (POC) pour tester et valider les concepts, la personnalisation de l’outil avec l’adaptation du framework de sécurité et la modification de l’interface utilisateur, ainsi que la mise en place d’une architecture de déploiement sécurisée sur Azure.

Pour l’avenir, plusieurs axes d’amélioration et de développement peuvent être envisagés. Il est essentiel de poursuivre l’optimisation de l’outil pour améliorer ses performances et sa convivialité, tout en adoptant une approche de développement en itération. Chaque itération permettra d'intégrer des retours d'expérience et d'ajuster les fonctionnalités en conséquence. L’extension des fonctionnalités pour couvrir un plus large éventail de services Microsoft sera également une priorité. De plus, il est crucial de renforcer les mesures de sécurité pour s’assurer que l’outil reste conforme aux standards les plus stricts. Enfin, la participation à des formations continues permettra de rester à jour avec les dernières évolutions technologiques, garantissant ainsi que le projet évolue de manière agile et réactive face aux changements du secteur.

Au terme de ce travail, je peux affirmer que ce stage a été plein d’intérêt et d'apprentissage. J'ai pu travailler sur de nouveaux logiciels et découvrir le déroulement de la vie professionnelle à travers des réunions avec mon encadrant et les autres membres de l’entreprise. J'ai appris comment une équipe travaille pour réussir un projet, ce qui m'a donné une idée des situations que je vais confronter dans un futur proche en tant que DevOps/Architecte Solutions Cloud. Les compétences et les expériences acquises seront des fondations solides pour ma future carrière professionnelle.
