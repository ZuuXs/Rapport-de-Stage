Ce document représente une étude synthétique du travail réalisé dans le
cadre de mon stage de troisième année. J'ai effectué mon stage au sein de
l'université Paris Est Créteil (\textbf{UPEC})  et de l'entreprise Exakis Nelite (\textbf{EXN}), collaboratrice de Microsoft. Ce stage a duré trois mois, de fin avril à fin juillet 2024. L'objectif principal de ce projet était de développer un outil ainsi qu'une interface web pour la gestion, la configuration et l'audit des environnements Microsoft et Azure cloud, souvent appelés "tenant" ensemble. La réalisation du projet s'est déroulée en trois étapes principales :
\begin{itemize}
	\item[•] Analyse détaillée des besoins ;
	\item[•] Conception des différentes composantes de l'application ;
	\item[•] Implémentation des objectifs définis.
\end{itemize}

J'ai donc eu l'opportunité d'exploiter une variété de technologies, outils et concepts qui ont grandement contribué à la réalisation de cette solution. Parmi ces ressources, plusieurs proviennent des cours que j'ai suivis au cours de l'année précédente, consolidant ainsi mes connaissances théoriques. En outre, j'ai découvert et appris à maîtriser de nouveaux outils durant ma formation au sein de l'entreprise, ce qui m'a permis d'acquérir des compétences pratiques supplémentaires. Vous trouverez un récapitulatif détaillé de ces technologies et outils utilisés dans les annexes de ce document.

Le présent rapport est structuré de manière à offrir une vue d'ensemble claire et détaillée de l'ensemble du projet. Le chapitre I est consacré à la présentation de l'organisme d'accueil, y compris l’introduction à l’équipe et l’entreprise, ainsi que la définition des objectifs du stage. Le chapitre II se concentre sur le cadrage fonctionnel et la présentation du projet, détaillant le benchmark des solutions disponibles, et l'étude conceptuelle des besoins et des normes de sécurité. Le chapitre III aborde le cadrage technique, où sont élaborées les différentes composantes techniques de la solution envisagée, incluant l'analyse de l'existant, les premières utilisations et la personnalisation de la solution MONKEY365. Enfin, le chapitre IV se consacre à la partie réalisation, où je vais aborder mes réalisations concrètes pendant le stage, telles que l'obtention de certifications, l'introduction au DevOps, la construction d'applications avec Azure DevOps, la participation à un événement Microsoft sur les nouvelles technologies IA, et le développement et déploiement d'un site web servant de portfolio personnel, mettant en lumière les contributions spécifiques que j'ai apportées au projet.

\textbf	{\\Mots clés :\\ Gestion, Configuration, Audit, Microsoft, Azure cloud, Tenant, Conception, Annexe,  Technologie.}
